\documentclass[10pt,a4paper,oneside]{article}
\usepackage[utf8]{inputenc}
\usepackage[english,russian]{babel}
\usepackage{amsmath}
\usepackage{amsthm}
\usepackage{amssymb}
\usepackage{enumerate}
\usepackage{stmaryrd}
\usepackage{cmll}
\usepackage{mathrsfs}
\usepackage[left=2cm,right=2cm,top=2cm,bottom=2cm,bindingoffset=0cm]{geometry}
\usepackage{proof}
\usepackage{tikz}
\usetikzlibrary{arrows,backgrounds,patterns,matrix,shapes,fit,calc,shadows,plotmarks}
\begin{document}

\begin{center}{\Large\textsc{\textbf{Теоретические (``малые'') домашние задания}}}\\
             \it Математическая логика, ИТМО, М3234-М3239, весна 2019 года\end{center}

\section*{Домашнее задание №1: <<знакомство с исчислением высказываний>>}

\begin{enumerate}

\item Расставьте скобки:
\begin{enumerate}
\item $\alpha\to\alpha\to\neg\beta\vee\beta\with\neg\alpha\vee\neg\beta\to\alpha\with\alpha\to\alpha\vee\beta\vee\beta$
\end{enumerate}

\item Покажите следующие утверждения, построив полный вывод (в частности, если пользуетесь теоремой о дедукции --- 
раскройте все преобразования):
\begin{enumerate}
\item $\alpha\vee\beta \vdash \neg(\neg\alpha\with\neg\beta)$
\item $\alpha\with\beta \vdash \neg(\neg\alpha\vee\neg\beta)$
\item $\alpha\to\beta\to\gamma \vdash \alpha\with\beta\to\gamma$
\item $\alpha\with\beta\to\gamma \vdash \alpha\to\beta\to\gamma$
\item $\alpha,\neg\alpha \vdash \beta$
\end{enumerate}

\item Покажите следующие утверждения, построив полный вывод (за полный
ответ будет считаться доказательство пяти утверждений из списка):
\begin{enumerate}
\item $\gamma \vdash \alpha\to\gamma$
\item $\alpha,\beta \vdash \alpha\&\beta$
\item $\neg\alpha,\beta \vdash \neg(\alpha\&\beta)$
\item $\alpha,\neg\beta \vdash \neg(\alpha\&\beta)$
\item $\neg\alpha,\neg\beta \vdash \neg(\alpha\&\beta)$
\item $\alpha,\beta \vdash \alpha\vee\beta$
\item $\neg\alpha,\beta \vdash \alpha\vee\beta$
\item $\alpha,\neg\beta \vdash \alpha\vee\beta$
\item $\neg\alpha,\neg\beta \vdash \neg(\alpha\vee\beta)$
\item $\alpha,\beta \vdash \alpha\rightarrow\beta$
\item $\alpha,\neg\beta \vdash \neg(\alpha\rightarrow\beta)$
\item $\neg\alpha,\beta \vdash \alpha\rightarrow\beta$
\item $\neg\alpha,\neg\beta \vdash \alpha\rightarrow\beta$
\item $\neg\alpha \vdash \neg\alpha$
\item $\alpha \vdash \neg\neg\alpha$
\end{enumerate}
\end{enumerate}

\section*{Домашнее задание №2: <<исчисление высказываний>>}

\begin{enumerate}

\item (Теоремы о корректности и полноте) Пусть $\Gamma$ --- какой-то список
высказываний и пусть $\alpha$ --- высказывание.

\begin{enumerate}
\item Покажите, что $\Gamma\vdash\alpha$ влечёт $\Gamma\models\alpha$.
\item Покажите, что $\Gamma\models\alpha$ влечёт $\Gamma\vdash\alpha$.
\end{enumerate}

\item (Теорема Гливенко) Рассмотрим исчисление высказываний, в котором 10 схема аксиом 
(аксиома снятия двойного отрицания)
$$\neg\neg\alpha\rightarrow\alpha$$
заменена на следующую:
$$\alpha\rightarrow\neg\alpha\rightarrow\beta$$
Такой вариант исчисления высказываний назовём интуиционистским.
Будем писать $\Gamma \vdash_{\texttt{и}} \alpha$, если существует
вывод формулы $\alpha$ из гипотез $\Gamma$ в интуиционистском исчислении
высказываний. Если же вывод производится в классическом исчислении
(изученном на 1 и 2 занятиях), будем указывать это как $\Gamma \vdash_{\texttt{к}} \alpha$.

\begin{enumerate}
\item Покажите, что если $\Gamma \vdash_{\texttt{и}} \alpha$, то 
$\Gamma \vdash_{\texttt{к}} \alpha$.
\item Покажите, что если $\alpha$ --- аксиома ($1\dots 9$ схемы), то 
$\vdash_{\texttt{и}} \neg\neg\alpha$.
\item Покажите, что $\vdash_{\texttt{и}} \neg\neg(\neg\neg\alpha\rightarrow\alpha)$.
\item Покажите, что если $\vdash_{\texttt{и}} \neg\neg\alpha$ и $\vdash_{\texttt{и}} \neg\neg(\alpha\rightarrow\beta)$,
то $\vdash_{\texttt{и}} \neg\neg\beta$.
\item Покажите, что если $\vdash_{\texttt{к}} \alpha$, то $\vdash_{\texttt{и}} \neg\neg\alpha$ (теорема Гливенко).
\item Покажите, что если $\Gamma \vdash_{\texttt{к}} \alpha$, то $\Gamma \vdash_{\texttt{и}} \neg\neg\alpha$.
\item Назовём (классическое или интуиционистское) исчисление \emph{противоречивым}, 
если для любой формулы $\alpha$ выполнено $\vdash \alpha$.
Покажите, что формула $\alpha$ исчисления, такая, что $\vdash \alpha$ и $\vdash\neg\alpha$, 
существует тогда и только тогда, когда исчисление противоречиво.
\item Покажите, что если классическое исчисление высказываний противоречиво, то противоречиво и интуиционистское
исчисление высказываний.
\end{enumerate}

\end{enumerate}

\section*{Домашнее задание №3: <<общая топология>>}

Назовём топологическим пространством упорядоченную пару $\langle X, \Omega \rangle$, 
где $X$ --- некоторое множество, а $\Omega \subseteq \mathcal{P}(X)$ --- множество каких-то 
подмножеств $X$.
Множество $X$ мы назовём \emph{носителем} топологии (также можем назвать его топологическим 
пространством), а $\Omega$ --- \emph{топологией}.
Элементы множества $\Omega$ мы будем называть \emph{открытыми} множествами.
При этом пара должна удовлетворять следующим свойствам \emph{(аксиомам топологического пространства)}:

\begin{enumerate}
\item $\varnothing \in \Omega, X \in \Omega$ (пустое множество и всё пространство открыты);
\item Если $\{A_i\}$, $A_i \in \Omega$ --- некоторое семейство элементов $\Omega$,
то $\bigcup_i A_i \in \Omega$ (объединение произвольного семейства открытых множеств открыто);
\item Если $A_1, A_2, \dots, A_n$, $A_i \in \Omega$ --- конечное множество открытых множеств, 
то его пересечение также открыто: $A_1 \cap A_2 \cap \dots \cap A_n \in \Omega$
\end{enumerate}

Решите следующие задачи:

\begin{enumerate}
\item Задачи на определение пространства:
\begin{enumerate}
\item Покажите, что при $X = \{0,1\}$, $\Omega = \{\varnothing, \{0\}, \{0,1\}\}$ пара $\langle X, \Omega \rangle$
является топологическим пространством.
\item Покажите, что если $X$ --- непустое множество, то пара $\langle X, \{ \varnothing, X \}\rangle$ является
топологическим пространством.
\item Предложите примеры как минимум двух множеств $\Omega \subseteq \mathcal{P}\{0,1\}$, для которых
$\langle \{0,1\}, \Omega\rangle$ --- не топологическое пространство.
\item Для каждой аксиомы топологического пространства приведите примеры таких пар 
$\langle X, \Omega\rangle$, в которых бы аксиома не была выполнена.
\item Для $X = \mathbb{R}^n$ и $\Omega$, содержащего все открытые множества (в смысле метрического
определения, данного на мат. анализе), покажите, что $\langle X, \Omega \rangle$ является 
топологическим пространством.
\end{enumerate}

\item Про каждое определение ниже покажите, что оно действительно задаёт топологическое пространство.
\begin{enumerate}
\item $X = \mathbb{R}$, $\Omega = \{(x,+\infty) | x \in \mathbb{R}\} \cup \{\varnothing\}$ (топология стрелки)
\item $X \ne \varnothing$, $\Omega = \mathcal{P}(X)$ (дискретная топология)
\item $X = \mathbb{R}$, $\Omega = \{ A | A \subseteq\mathbb{R}, \mathbb{R}\setminus A\,\textrm{--- конечно}\}$
--- множество всех множеств, дополнение которых конечно (топология Зарисского)
\item $X$ --- некоторое дерево, а открытыми множествами на нём назовём все множества, которые
содержат узел вместе со всеми своими потомками:
$A \in \Omega$ тогда и только тогда, когда если $a \in A$ и $a \ge b$, то $b \in A$.
\end{enumerate}

\item \emph{Замкнутым} множеством назовём множество, дополнение которого открыто. 
\begin{enumerate}
\item Покажите, что пересечение произвольного семейство замкнутых множеств --- замкнуто.
\item Пусть $A$ --- замкнутое, а $B$ --- открытое множество в некотором пространстве.
Что вы можете сказать про замкнутость или открытость $B\setminus A$ и $A \setminus B$?
\end{enumerate}

\item Определим операции <<взятие внутренности>> и <<взятие замыкания>>, покажите корректность этих определений
(т.е. что определяемый объект существует):
\begin{enumerate}
\item Для множества $A$ внутренностью $A^\circ$ назовём максимальное открытое множество, 
что $A^\circ \subseteq A$. 
\item Для множества $A$ замыканием $\overline{A}$ назовём минимальное замкнутое множество, содержащее $A$.
\end{enumerate}

\item Найдите $[0,1]^\circ$ и $\overline{[0,1]}$ в первых трёх топологиях из п. 2 
(если взять в качестве носителя $\mathbb{R}$)?
\item Найдите $\{0\}^\circ$ и $\overline{\{0\}}$ в первых трёх топологиях из п. 2 
(если взять в качестве носителя $\mathbb{R}$)?

\end{enumerate}

\section*{Домашнее задание №4: <<решётки, псевдобулевы и булевы алгебры>>}

\begin{enumerate}

\item Пусть задана некоторая решётка, в которой задано псевдодополнение.
Докажите, что эта решётка является дистрибутивной.

\item Пусть задана дистрибутивная решётка. Покажите, что в ней для любых элементов 
$a,b,c$ выполнено $(a+b)\cdot c = a\cdot c + b \cdot c$.
А будет ли выполнено $(a+b)\cdot c = a\cdot c \rightarrow b \cdot c$?

\item Покажите, что если в решётке есть \emph{диамант} или \emph{пентагон}
(то есть, найдутся 5 элементов указанным образом упорядоченных, среди которых есть две или 
три пары несравнимых), то решётка не является дистрибутивной:
\begin{center}\tikz{
\node at (1,2)   (A) {A};
\node at (0,0.5) (B) {B};
\node at (1,0.5) (C) {C};
\node at (2,0.5) (D) {D};
\node at (1,-1)  (E) {E};
\draw[->] (A) to (B); \draw[->] (B) to (E);
\draw[->] (A) to (C); \draw[->] (C) to (E);
\draw[->] (A) to (D); \draw[->] (D) to (E);

\node at (5,2)   (A1) {A};
\node at (4,0.5) (B1) {B};
\node at (6,1)   (C1) {C};
\node at (6,0)   (D1) {D};
\node at (5,-1)  (E1) {E};
\draw[->] (A1) to (B1); \draw[->] (B1) to (E1);
\draw[->] (A1) to (C1); \draw[->] (C1) to (D1); \draw[->] (D1) to (E1);
}\end{center}

\item Предложите пример дистрибутивной, но не импликативной решётки.

\item Докажите, что в импликативной решётке при любых значениях $a$, $b$ и $c$ 
выполнены следующие утверждения:

\begin{enumerate}
\item Из $a \sqsubseteq b$ следует $b\to c \sqsubseteq a\to c$ и $c\to a \sqsubseteq c \to b$;
\item Из $a \sqsubseteq b \to c$ следует $a \cdot b \sqsubseteq c$;
\item $a \sqsubseteq b$ выполнено тогда и только тогда, когда $a \to b = 1$;
\item $b \sqsubseteq a \rightarrow b$;
\item $a \rightarrow b \sqsubseteq ((a \rightarrow (b \rightarrow c)) \rightarrow (a \rightarrow c))$;
\item $a \sqsubseteq b \rightarrow a \cdot b$;
\item $a \rightarrow c \sqsubseteq (b \rightarrow c) \rightarrow (a + b \rightarrow c)$
\end{enumerate}

\item Пусть заданы некоторая алгебра Гейтинга $\langle H, \sqsubseteq\rangle$
и переменные $A$, $B$, $C$ со значениями $a$, $b$, $c$ ($a,b,c \in H$). Покажите, что:
\begin{enumerate}
\item[(a-i)] Если $\phi$ --- схема аксиом 1--9, то при подстановке переменных $A$, $B$, $C$ вместо 
вместо метапеременных при любых $a$, $b$, $c$ будет выполнено $\llbracket\phi\rrbracket = \texttt{И}$;
\item[(j)] Аналогично, будет выполнено 
$\llbracket\alpha\rightarrow\neg\alpha\rightarrow\beta\rrbracket = \texttt{И}$;
\item[(k)] Если заданная алгебра Гейтинга --- булева, то тогда выполнено и
$\llbracket\alpha\rightarrow\neg\neg\alpha\rrbracket = \texttt{И}$ и 
$\llbracket\alpha\vee\neg\alpha\rrbracket = \texttt{И}$.
\item[(l)] Пусть $\phi$ и $\phi\rightarrow\tau$ --- некоторые истинные высказывания в указанной
алгебре при указанных значениях переменных. Тогда $\tau$ --- тоже истинное высказывание
\end{enumerate}

\item На основании предыдущего пункта покажите, что алгебра Гейтинга корректна как модель ИИВ,
и что булева алгебра корректна как модель ИВ.

\item Про следующие высказывания определите, являются ли они доказуемыми в ИИВ:
\begin{enumerate}
\item $((P\to Q)\to P)\to P$ (закон Пирса);
\item $(\neg P\to Q)\vee(P \to\neg Q)$;
\item $(P \to \neg Q) \rightarrow (Q \to \neg P)$;
\item $P \rightarrow \neg\neg P$;
\item $\neg\neg P \vee \neg\neg\neg P$;
\end{enumerate}

\end{enumerate}

\section*{Домашнее задание №5: <<Гёделевы алгебры, модели Крипке>>}

\begin{enumerate}
\item Ещё немного про решётки. Будем говорить, что решётка содержит \emph{диамант} или \emph{пентагон},
если найдутся 5 элементов указанным на диаграмме образом упорядоченных.
При этом, если $p + q = r$ или $p \cdot q = r$ на данной диаграмме, то это же свойство выполнено 
и в исходной решётке.

\begin{center}\tikz{
\node at (1,2)   (A) {A};
\node at (0,0.5) (B) {B};
\node at (1,0.5) (C) {C};
\node at (2,0.5) (D) {D};
\node at (1,-1)  (E) {E};
\draw[->] (A) to (B); \draw[->] (B) to (E);
\draw[->] (A) to (C); \draw[->] (C) to (E);
\draw[->] (A) to (D); \draw[->] (D) to (E);

\node at (5,2)   (A1) {A};
\node at (4,0.5) (B1) {B};
\node at (6,1)   (C1) {C};
\node at (6,0)   (D1) {D};
\node at (5,-1)  (E1) {E};
\draw[->] (A1) to (B1); \draw[->] (B1) to (E1);
\draw[->] (A1) to (C1); \draw[->] (C1) to (D1); \draw[->] (D1) to (E1);
}\end{center}
\begin{enumerate}

\item Назовём решётку \emph{модулярной}, если при всяких $x$ и $z$, таких, что $z \sqsubseteq x$,
выполнено $(x \cdot y) + z = x \cdot (y + z)$. Покажите, что решётка является модулярной
тогда и только тогда, когда не содержит пентагонов.

\item Рассмотрим модулярную решётку: покажите, что она дистрибутивна тогда и только тогда,
когда не содержит диамантов.
\end{enumerate}

\item Покажите, что $(\approx)$ является отношением эквивалентности.
На основании этого покажите, что определение $[\alpha]_\approx \sqsubseteq [\beta]_\approx$ корректно
(не зависит от выбора конкретных представителей класса эквивалентности).

\item Пусть $A$ --- алгебра Гейтинга. Покажите, что $\Gamma(A)$ --- тоже алгебра Гейтинга.

\item Пусть задана алгебра Гейтинга $A$:
\begin{center}\tikz{
\node at (1,2)   (A) {P};
\node at (0,1) (B) {Q};
\node at (2,1) (D) {R};
\node at (1,0)  (E) {S};
\draw[->] (A) to (B); \draw[->] (B) to (E);
\draw[->] (A) to (D); \draw[->] (D) to (E);
}\end{center}
Постройте $\Gamma(A)$.

\item Можно ли для алгебры $\Gamma(\mathbb{R})$ построить топологию, порождающую данную
алгебру? Вам нужно определить какой-то новый носитель и открытые множества для нём --- или указать,
что это невозможно.

\item Могло сложиться впечатление, что $\mathscr{L}$ и $\Gamma(\mathscr{L})$ почти ничем не отличаются.
В связи с этим давайте немного изучим данный вопрос:

\begin{enumerate}
\item Мы выяснили, что алгебра Линденбаума --- полная модель ИИВ. 
А справедливо ли это для $\Gamma(\mathscr{L})$ --- существует ли формула $\alpha$,
общезначимая в $\Gamma(\mathscr{L})$, но недоказуемая?
\item Приведите пример неатомарной формулы $\alpha$ и такой оценки переменных, что 
$\llbracket\alpha\rrbracket_{\Gamma(\mathscr{L})} = \omega$. 
\item Мы можем построить аналог алгебры Линденбаума для классического ИВ, а потом применить к ней
операцию <<гёделевизации>>. Но если так получится доказать свойство дизъюнктивности для классической логики,
то мы найдём противоречие в логике. Какое противоречие мы получим и какой переход в наших рассуждениях 
не получится сделать по аналогии?
\end{enumerate}

\item Рассмотрим два множества, $a = (-\infty,1)$ и $b = (0,\infty)$. 
Пусть $\llbracket A \rrbracket = a$ и $\llbracket B \rrbracket = b$. Понятно, что 
$\llbracket A \vee B \rrbracket_\mathbb{R} = 1$. 
Однако, ни $A$, ни $B$ не истинны --- не закралась ли где ошибка в теорему о 
дизъюнктивности ИИВ? 

\item {\bfseries Модели Крипке.} Рассмотрим некоторый ориентированный граф без циклов (без потери общности можем
взять дерево вместо такого графа). Узлы назовём \emph{мирами} и пронумеруем натуральными
числами: $W = \{ W_1, W_2, \dots, W_n\}$. Будем писать $W_i \preceq W_j$, если существует путь
из $W_i$ в $W_j$. Понятно, что $W_i \preceq W_i$.

Каждому узлу сопоставим множество \emph{вынужденных} переменных ИИВ и будем писать 
$W_i \Vdash A_k$, если переменная $A_k$ вынуждена в мире $W_i$. При этом, если $W_i \preceq W_j$,
то всегда должно быть выполнено и $W_j \Vdash A_k$ (знание, полученное нами, не исчезает
в последующих мирах).

Обобщим отношение вынужденности на случай произвольной формулы:
\begin{itemize}
\item Если $W_i \Vdash \alpha$ и $W_i \Vdash \beta$, то $W_i \Vdash \alpha\with\beta$;
\item Если $W_i \Vdash \alpha$ и $W_i \Vdash \beta$, то $W_i \Vdash \alpha\vee\beta$;
\item Если в любом мире $W_k: W_i \preceq W_k$ выполнено, что из $W_k \Vdash \alpha$ следует $W_k \Vdash \beta$, 
то $W_i \Vdash \alpha\rightarrow\beta$;
\item Если ни в каком мире $W_k: W_i \preceq W_k$ не выполнено $\alpha$, то $W_i \Vdash \neg\alpha$.
\end{itemize}

Так определённую упорядоченную тройку $\langle W, (\preceq), (\Vdash) \rangle$ --- множество миров, отношение 
порядка на мирах и отношение вынужденности --- назовём моделью Крипке.
Будем говорить, что формула $\alpha$ вынуждается моделью (или является истинной в данной модели),
если $W_i \Vdash \alpha$ в любом мире $W_i$. Будем записывать это как $\Vdash \alpha$.

\begin{enumerate}
\item Построим пример модели, опровергающей формулу $P \vee\neg P$
(деревья в моделях Крипке у нас будут расти вправо):
\begin{center}\tikz{
\node at (0,0)   (A) {$W_1$};
\node at (2,1) (B) {$W_2$};
\node at (2,-1) (C) {$W_3$};
\draw[->] (A) to (B); 
\draw[->] (A) to (C); 
}\end{center}

В данной модели переменная $P$ вынуждена только в мире $W_2$.

Укажите все узлы, в которых вынуждено $P$, $\neg P$, $P \vee\neg P$ и сделайте вывод о вынужденности
закона исключённого третьего в данной модели.

\item Постройте модель, опровергающую формулу $((P\rightarrow Q)\rightarrow P)\rightarrow P$.

\item Покажите, что любая модель Крипке обладает свойством: для любых $W_i, W_j, \alpha$, 
если $W_i \preceq W_j$ и $W_i \Vdash \alpha$, то $W_j \Vdash \alpha$.

\item Покажите, что по любой модели Крипке $K$ можно построить такую алгебру Гейтинга $H$, что
$\Vdash_K \alpha$ тогда и только тогда, когда $\llbracket\alpha\rrbracket_H = 1_H$. 
Покажите из этого, что любая модель Крипке --- действительно модель ИИВ.

\item Предложите формулу, глубина опровергающей модели для которой (если её рассматривать как 
дерево) не может быть меньше 2. Можете ли предложить соответствующую конструкцию для произвольной 
глубины $n$?
\end{enumerate}

\item Теорема о нетабличности интуиционистской логики.
\begin{enumerate}
\item Рассмотрим следующее утверждение $(A \to B)\vee(B\to C)\vee(C \to A)$: 
покажите, что это утверждение верно в классической логике, но недоказуемо в интуиционистской. 
Интуитивно недоказуемость в интуиционистской логике очевидна: пусть $A$ --- сегодня дождь, 
$B$ --- сегодня мороз $-30^\circ$ по Цельсию,  $C$ --- сегодня понедельник. 
У нас нет никаких конструктивных способов показать из одного утверждения другое.

\item Обозначим за $\rho_n$ следующее утверждение:
$$\bigvee_{i \ne j; 0 < i,j \le n} (A_i \to A_j)$$ 
Покажите, что для любой табличной модели ИИВ $T$ найдётся такое $n$,
что $\llbracket \rho_n \rrbracket_T = \texttt{И}$.

\item Покажите, что $\nvdash \rho_n$ в ИИВ ни при каком $n > 1$. Как из этого показать, 
что никакая табличная модель ИИВ не является полной?
\end{enumerate}

\end{enumerate}

\section*{Домашнее задание №6: <<Исчисление предикатов>>}

\begin{enumerate}
\item Новые аксиомы и правила вывода для исчисления предикатов имеют ограничения
(требования свободы для подстановки и отсутствия свободных вхождений). 
Если эти требования будут нарушены, исчисление станет некорректным.
Данный факт можно показать, построив соответствующие опровергающие
оценки для каких-то доказуемых формул. Постройте соответствующие формулы,
доказательства и оценки.

\item Докажите следующие формулы в исчислении предикатов:
\begin{enumerate}
\item $\forall a.\phi\rightarrow \phi$
\item $(\forall a.\phi)\rightarrow (\exists a.\phi)$
\item $(\forall a.\forall a.\phi) \rightarrow (\forall a.\phi)$
\item $(\forall a.\phi) \rightarrow (\neg \exists a.\neg \phi)$ 
\item $(\exists a.\phi) \rightarrow (\neg \forall a.\neg \phi)$
\item $(\forall a.\neg\phi) \rightarrow (\neg \exists a.\phi)$ 
\item $(\exists a.\neg\phi) \rightarrow (\neg \forall a.\phi)$
\end{enumerate}

\item Опровергните формулы $\phi\rightarrow\forall a. \phi$ и $(\exists a.\phi)\rightarrow (\forall a.\phi)$

\item Докажите теорему о дедукции для исчисления предикатов:
\begin{enumerate}
\item Если $\Gamma, \alpha \vdash \beta\rightarrow\forall a.\gamma$, то
$\Gamma\vdash\alpha\rightarrow\beta\rightarrow\forall a.\gamma$ (если $a$ не входит
свободно в $\alpha$).
\item Если $\Gamma, \alpha \vdash (\exists a.\gamma)\rightarrow\beta$, то
$\Gamma\vdash\alpha\rightarrow(\exists a.\gamma)\rightarrow\beta$ (если $a$ не входит
свободно в $\alpha$).
\item Сведите всё вместе и постройте общее доказательство теоремы.
\end{enumerate}

\item Рассмотрим формулу $\alpha$ с двумя свободными переменными $a$ и $b$.
Определите, какие из сочетаний кванторов выводятся из каких:
\begin{enumerate}
\item $\forall a.\forall b.\alpha$, $\forall b.\forall a.\alpha$
\item $\exists a.\exists b.\alpha$, $\exists b.\exists a.\alpha$
\item $\forall a.\forall b.\alpha$, $\forall a.\exists b.\alpha$, $\exists a.\forall b.\alpha$, $\exists a.\exists b.\alpha$
\item $\forall a.\exists b.\alpha$, $\exists b.\forall a.\alpha$
\end{enumerate}

\end{enumerate}

\section*{Домашнее задание №7: <<Исчисление предикатов>>}

\begin{enumerate}
\item Пусть дано непротиворечивое множество замкнутых формул $\Gamma$, и пусть
дана формула $\alpha$. Покажите, что как минимум либо $\Gamma \cup \alpha$, либо
$\Gamma \cup \neg\alpha$ непротиворечиво.

\item Пусть $D$ --- предметное множество, и оно состоит из строк. Пусть $\Gamma$ --- некоторое
полное непротиворечивое множество замкнутых бескванторных формул.
Пусть $$\llbracket f_i (\theta_1, \theta_2, \dots \theta_k)\rrbracket = 
\textrm{<<}f_i (\textrm{>>} ++ \llbracket \theta_1 \rrbracket ++ 
\llbracket \theta_2 \rrbracket ++ \dots ++ \llbracket \theta_k \rrbracket ++ \textrm{<<})\textrm{>>}$$

Пусть $$P_i (\theta_1, \theta_2, \dots, \theta_k) = 
   \left\{ \begin{array}\texttt{И,}& \textrm{если } P_i (\theta_1, \theta_2, \dots, \theta_k) \in \Gamma\\
           \texttt{Л,} & \textrm{иначе}\end{array}\right.$$

Покажите тогда, что $\psi \in \Gamma$ тогда и только тогда, когда $\llbracket\psi\rrbracket = \texttt{И}$.
Для этого:
\begin{enumerate}
\item покажите, что $\alpha\with\beta \in \Gamma$ тогда и только тогда, когда $\alpha\in\Gamma$ и $\beta\in\Gamma$;
\item покажите, что $\alpha\with\beta \in \Gamma$ тогда и только тогда, когда $\alpha\in\Gamma$, либо $\beta\in\Gamma$,
либо выполнено оба утверждения;
\item покажите, что $\alpha\in\Gamma$ тогда и только тогда, когда $\neg\alpha\notin\Gamma$.
\item покажите, что $\alpha\rightarrow\beta \in \Gamma$ тогда и только тогда, когда 
либо $\neg\alpha\in\Gamma$, либо одновременно $\alpha\in\Gamma$ и $\beta\in\Gamma$.
\item при помощи сформулированных выше вспомогательных утверждений докажите требуемое утверждение.
\end{enumerate}

\item Формализация понятий свободных переменных, свободы для подстановки, замены переменных.

Рассмотрим множество $FV$ (свободных переменных) для формул:

$$
FV(\psi) = \left\{ \begin{array}{rl}
  \bigcup_i FV(\theta_i),  &\mbox{ если $\psi$ --- $P(\theta_1,\theta_2,\dots,\theta_n)$ } \\
  FV(\alpha) \cup FV(\beta),  &\mbox{ если $\psi$ имеет вид $\alpha\with\beta$, $\alpha\vee\beta$, $\alpha\rightarrow\beta$, $\neg \alpha$ }\\
  FV(\varphi) \setminus \{x\},  &\mbox{ если $\alpha$ имеет вид $\forall x.\varphi$ или $\exists x.\varphi$ }
       \end{array} \right.
$$

и для термов:

$$
FV(\theta) = \left\{ \begin{array}{rl}
  x, &\mbox{ если 
  \bigcup_i FV(\theta_i),  &\mbox{ если $\psi$ --- $P(\theta_1,\theta_2,\dots,\theta_n)$ } \\
\end{array} \right.
$$


Рассмотрим операцию замены переменных, определим её для формул:

$$
\alpha [x := \theta] = \left\{ \begin{array}{rl}
  \alpha, &\mbox{ если $x \notin FV(\alpha)$}\\
  P(\theta_1 [x := \theta], \theta_2 [x := \theta], \dots \theta_k [x := \theta], &\mbox{ если $\alpha$ --- предикатный символ}\\
  (\psi [x := \beta]) \star (\varphi [x := \beta]), &\mbox{ если $\alpha$ имеет вид $\psi\star\varphi$ }\\
  \forall y.(\psi [x := \beta]),  &\mbox{ если $\alpha$ имеет вид $\forall y.\psi$ }
  \exists y.(\psi [x := \beta]),  &\mbox{ если $\alpha$ имеет вид $\exists y.\psi$ }
       \end{array} \right.
$$

и для термов:



\emph{Контекстом} подстановки $\alpha [x := B]$ назовем следующую функцию от
терма и заменяемой переменной:
$$
CX (A, x) = \left\{ \begin{array}{rl}
  \emptyset, &\mbox{ если $x \notin FV(A)$ или если $A$ --- переменная } \\
  CX (P, x) \cup CX (Q, x), &\mbox{ если $A$ имеет вид $P\ Q$ }\\
  CX (R, x) \cup \{y\},  &\mbox{ если $A$ имеет вид $\+y.R$ }
       \end{array} \right.
$$

\begin{enumerate}
\item Покажите, что $x \in FV(\alpha)$ тогда и только тогда, когда $x$ входит свободно в $\alpha$.
\item Продолжение следует...
\end{enumerate}


\end{enumerate}

\end{document}
