\documentclass[11pt,a4paper,oneside]{article}
\usepackage[utf8]{inputenc}
\usepackage[english,russian]{babel}
\usepackage{amssymb}
%\usepackage{amsmath}
%\usepackage{mathabx}
\usepackage{stmaryrd}
\usepackage[left=2cm,right=2cm,top=2cm,bottom=2cm,bindingoffset=0cm]{geometry}
%\usepackage{bnf}
\newcommand{\lit}[1]{\mbox{`\texttt{#1}'}}
\newcommand{\ntm}[1]{<\mbox{#1}>}
\begin{document}


\begin{center}
\begin{Large}{\bfseries Вопросы к коллоквиуму по курсу <<Математическая логика>>}\end{Large}\\
\vspace{1mm}
\begin{small} ИТМО, группы M3234..M3239\end{small}\\
\small 2, 3, 4 апреля 2018 г.
\end{center}

Порядок проведения коллоквиума: каждому будет задано два вопроса --- либо
определение (формулировка теоремы) из списка ниже, либо простая очевидная
задача в одно действие на понимание указанных понятий и определений
(\emph{пример:} приведите высказывание, недоказуемое в 
классическом исчислении высказываний). Если понятие прямо не перечислено
в списке, но необходимо для формулировки теоремы или другого понятия из 
списка (\emph{пример:} понятие нетабличности исчисления), то его определение 
также может быть задано в качестве вопроса.

Ответ на каждый вопрос оценивается в 0, 1 или 2 балла (в зависимости от полноты
ответа) — в табличке оценок эти баллы будут обозначены как \verb!-!, \verb!?!
или \verb!+!; всего за коллоквиум можно получить от 0 до 4 баллов.

Некоторые вопросы адресуются только учащимся групп М3238--М3239.
Такие вопросы особым образом помечены в списке.

\begin{itemize}
\item Топология: топологическое пространство,
открытое и замкнутое множество, внутренность и замыкание множества, топология стрелки,
дискретная топология, топология на частично упорядоченном множестве,
индуцированная топология на подпространстве, связность.
\item Исчисление высказываний: метапеременные, пропозициональные переменные,
высказывания, аксиомы, схемы аксиом, правило Modus Ponens, доказательство,
вывод из гипотез, доказуемость, множество истинностных значений, 
модель (оценка переменных), оценка высказывания, общезначимость, 
выполнимость, невыполнимость, следование, корректность, полнота, 
противоречивость; формулировка теорем о дедукции, о корректности и о полноте И.В.
\item Интуиционистское исчисление высказываний: 
закон исключённого третьего, закон снятия двойного отрицания,
закон Пирса, BHK-интер\-пре\-та\-ция логических связок, теорема Гливенко,
решётка, дистрибутивная решётка, импликативная решётка, 
алгебра Гейтинга, булева алгебра, 
Гёделева алгебра, операция $\Gamma(A)$, алгебра Линденбаума,
формулировка свойства дизъюнктивности И.И.В, формулировка свойства нетабличности И.И.В.

\emph{Дополнительные вопросы для М3238--М3239:}
модулярная решётка, что значит, что решётка содержит пентагон/диамант в качестве
подрешётки, модели Крипке, вынужденность, вложение моделей Крипке в алгебры Гейтинга 
(общая идея).
\item Исчисление предикатов:
предикатные и функциональные символы, константы и пропозициональные переменные,
свободные и связанные вхождения предметных переменных в формулу, 
свобода для подстановки, правила вывода для кванторов, аксиомы исчисления предикатов 
для кванторов, оценки и модели в исчислении предикатов, теорема о дедукции для 
исчисления предикатов (формулировка), теорема о корректности для исчисления 
предикатов (формулировка), полное множество (бескванторных) формул, 
модель для формулы, теорема Гёделя о полноте исчисления предикатов (формулировка),
следствие из теоремы Гёделя о полноте исчисления предикатов.
\end{itemize}

\end{document}